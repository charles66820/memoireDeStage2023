\section{Problématiques / Objectifs} % problématiques

\subsection{Sujet}

Le sujet est \emph{Interruptions en espace utilisateur pour le réseau BXI}.

\subsubsection{Nouveau mécanisme d'interruption en espace utilisateur}

Ce nouveau mécanisme qui permet de dérouler une interruption à partir de l'espace utilisateur et très récent.
Il est seulement disponible sur les processeur \intel{} Sapphire Rapids qui sont officiellement sortis le 10 janvier 2023.
% https://www.datacenterknowledge.com/intel/intel-launches-sapphire-rapids-after-4-delays-it-worth-it
La plupart des processeur ont été disponibles à la vente le 14 mars.
% https://www.intel.com/content/www/us/en/newsroom/news/4th-gen-intel-xeon-sprints-into-market.html
AMD n'a pas encore annoncé de support pour les interruption en espace utilisateur.

\subsubsection{contentions...} % contention mémoire autour du polling

Le fonctionnement actuelle de la progression des communications amène à des problèmes de contention mémoire car plusieurs threads peuvent chercher à lire / modifier la même zone mémoire.
Utiliser les interruptions permettrai de ne plus avoir ce problèmes car seul le thread concerné par une zone mémoire serai prévenu.

TODO: 

\subsection{Projet global}

Dans la plupart des autre domain les périphériques envoi une interruption ordinaire à l'application, par le bais des signaux ou d'un appel système bloquante, pour avertir d'un changement.
L'idée serai de faire la même chose en HPC grâce au interruption en espace utilisateur.
Le projet globale vise donc à faire progresser les communications entre plusieurs noeuds du réseau BXI sans faire de polling et en utilisant les interruptions en espace utilisateur.
Cela permettrai de réduire globalement le temps de calcul d'une application.
Pour ce faire la carte réseau BXI devra être capable de levé des interruptions en espace utilisateur.
Le fait de supprimer le polling et de réduire le temps de calcul permettra de diminué la consommation électrique.

% remonter les évènements réseau jusque la bibliothèque NewMadeleine avec un coût aussi faible que possible.

\subsubsection{Les objectifs}

Il y a plusieurs objectifs, les principaux sont :

\begin{itemize}
  \item La réduction du temps de calcul.
  \item utiliser des interruptions en espace utilisateur pour replacé le polling.
  \item Simplifier pour l'utilisateur le recouvrement des communications par du calcul pour qu'il n'est plus à ajouter des \code{MPI_Test} en pleins milieu de ces calcul.
  \item Amélioré la réactivité des communications sans qu'il y est besoin d'une unité de calcul dédier à l'attente active.
\end{itemize}

Ce stage est donc une premier étape de ce projet global.

\subsection{Objectifs du stage}

Le premier objectif est de défricher le fonctionnement des interruption en espace utilisateur à partir des éléments suivant :
\begin{itemize}
  \item Le manuel \intel{} de l'architectures 64 et IA-32 pour les développeurs logiciels
  \item La présentation du mécanisme de Sohil Mehta, ingénieur \intel{} qui à développé le patch noyau, qui est une diapositive associer à des discussion sur LWN.net /* TODO: cité */ \footnote{\url{https://lwn.net/Articles/869140/}/* TODO: déplacé */}
  \item le patch du noyau linux avec ces manuels. /* TODO: cité */
\end{itemize}
Le second objectif est de connaître les propriétés du mécanisme et de mesuré ça performance.
Le troisième objectif est de ne plus avoir à faire du polling que se sois dédié un tread ou utilisé les threads de façon opportunist, ne plus perdre du temps de calcule à poll et résoudre les problèmes de réactivité.
Pour cela on envisage l'intégration de ces interruption dans le driver de mémoire partagé (shm) de NewMadeleine.
Pour testé dans un premier temps le fonctionnement avec des communications entre processus (IPC\footnote{Inter Process Communication en anglais/* TODO: on garde pour certain acronyme? */}).
Pour intégrer les interruption dans les drivers NewMadeleine il faut également permettre la progression des communications à partir d'un handler d'interruption.
Le dernier objectif est de montré que l'utilisation d'interruption permet bien d'amélioré le recouvrement des communications par du calcul.

% cite : Intel\textsuperscript{\tiny{\textregistered}} 64 and IA-32 Architectures Software Developer's Manual

\subsection{La suite}

Les objects suivant du projet global seront traité dans une thèse qui fait suite au stage.
