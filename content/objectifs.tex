\section{Problématiques / Objectifs} % problématiques

\subsection{Sujet}

\subsubsection{Nouveau mécanisme d'interruption en espace utilisateur}

\begin{itemize}
  \item très récent...
  \item sur les dernier CPU Inter Sapphire Rapide...
  \item 
\end{itemize}

\subsubsection{contentions...}

\subsection{Ne plus faire de Polling}

\begin{itemize}
  \item utiliser des interruption en HPC comme dans les autre domain
  \item quand un périphériques a besoin de remonté une information il n'y a plus besoin de polling
\end{itemize}

\subsection{Objectifs}

\subsubsection{Objectifs global}

\begin{itemize}
  \item L'objectifs à terme est de faire progressé les communications entre 2 noeuds sans utiliser du Polling
  \item Donc permettre au carte réseau BXI de d'utiliser les interruption en espace utilisateur.
  \item Réduire globalement le temps de calcul. (Feedback Mathieu)
  \item 
\end{itemize}

\subsubsection{Objectif du stage}

\begin{itemize}
  \item Comprendre le fonctionnement de uintr (à partir du manuel du CPU, des manuel du patch et des quelque document disponible)
  \item Connaître leurs propriétés.
  \item Mesurer leur performance.
  \item intégré le mécanisme dans le driver de mémoire partagé (shm) de NewMadeleine, pour des communication entre processus (ipc)
  \item faire progresser les communication à partir d'un handler d'interruption
  \item évité les problèmes du polling :
  \begin{itemize}
    \item réactivité
    \item mobilisation d'une ressource ou truffé le code de MPI_Test (simplifie la vie à l'utilisateur)
    \item ...
  \end{itemize}
\end{itemize}
