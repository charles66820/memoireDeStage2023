\section{Problématiques / Objectifs} % problématiques

\subsection{Sujet}

Le sujet est \emph{Interruptions en espace utilisateur pour le réseau BXI} \cite{internshipSubject}.

\subsubsection{Nouveau mécanisme d'interruption en espace utilisateur}

Ce nouveau mécanisme qui permet de dérouler une interruption à partir de l'espace utilisateur est très récent.
Il est seulement disponible sur les CPU \intel{} Sapphire Rapids qui sont officiellement sortis le 10 janvier 2023.
\footnote{\url{https://www.datacenterknowledge.com/intel/intel-launches-sapphire-rapids-after-4-delays-it-worth-it}}
La plupart des CPU ont été disponibles à la vente le 14 mars.
\footnote{\url{https://www.intel.com/content/www/us/en/newsroom/news/4th-gen-intel-xeon-sprints-into-market.html}}
\emph{AMD} n'a pas encore annoncé de support pour les interruptions en espace utilisateur.

\subsection{Projet global}

Dans la plupart des autres domaines, les périphériques envoient une interruption ordinaire à l'application, par le biais des signaux ou d'un appel système bloquant, pour avertir d'un changement.
L'idée serait de faire la même chose en HPC grâce aux interruptions en espace utilisateur.
Le projet global vise donc à faire progresser les communications entre plusieurs nœuds du réseau \emph{BXI} sans faire de polling et en utilisant les interruptions en espace utilisateur.
Cela permettrait de réduire globalement le temps de calcul d'une application.
Pour ce faire, la carte réseau \emph{BXI} devra être capable de lever des interruptions en espace utilisateur.
Le fait de supprimer le polling et de réduire le temps de calcul permettra de diminuer la consommation électrique.

% remonter les évènements réseau jusque la bibliothèque NewMadeleine avec un coût aussi faible que possible.

\subsubsection{Les objectifs}

Il y a plusieurs objectifs, les principaux sont les suivants :

\begin{itemize}
  \item La réduction du temps de calcul.
  \item Utiliser des interruptions en espace utilisateur pour remplacer le polling.
  \item Simplifier pour l'utilisateur le recouvrement des communications par du calcul, afin qu'il n'ait plus besoin d'ajouter des \code{MPI_Test} en plein milieu des calculs.
  \item Améliorer la réactivité des communications sans avoir besoin d'une unité de calcul dédiée à l'attente active.
\end{itemize}

Ce stage est donc une première étape de ce projet global.

\subsection{Objectifs du stage}

Le premier objectif est de défricher le fonctionnement des interruptions en espace utilisateur à partir des éléments suivants :
\begin{itemize}
  \item Le manuel \intel{} de l'architecture 64 et IA-32 pour les développeurs logiciels \cite{intelSoftwareDevMan}.
  \item La présentation du mécanisme de \emph{Sohil Mehta}, ingénieur chez \intel{}, qui a développé le patch noyau. Cette présentation est une diapositive associée à des discussions sur LWN.net \cite{uintrLWN}.
  \item Le patch du noyau Linux \cite{intelUintrLinuxKernel} avec ses manuels \cite{intelUintrLinuxKernelMan}.
\end{itemize}

Le second objectif est de connaître les propriétés du mécanisme et de mesurer sa performance.
Le troisième objectif est de ne plus avoir à faire du polling,
que ce soit en dédiant un thread ou en utilisant les threads de façon opportuniste,
afin de ne plus perdre de temps de calcul à poll et de résoudre les problèmes de réactivité.

Le troisième objectif est de montrer que l'utilisation des \uintr{} dans les communications HPC est possible.
Pour cela, on se place dans un cadre simplifié :

\begin{itemize}
  \item On se concentre sur les communications entre processus (IPC, Inter-Process Communication en anglais).
  Pour cela, on va utiliser la mémoire partagée (shm) pour communiquer au lieu du réseau \emph{BXI}.
  \item On se limite aux cas généraux des communications (donc pas de gros messages, pas de communications multi-thread, etc).
\end{itemize}

On envisage donc l'intégration de ces interruptions dans le driver de mémoire partagée (shm) de NewMadeleine,
ce qui nous permet une utilisation réelle pour voir ce que donnerait une version complète.

Pour intégrer les interruptions dans les drivers de \emph{NewMadeleine}, il faut également permettre la progression des communications à partir d'un handler d'interruption.
Le dernier objectif est de montrer que l'utilisation d'interruption permet bien d'améliorer le recouvrement des communications par du calcul.

\subsection{La suite}

Les objectifs suivants du projet global seront traités dans une thèse qui fait suite au stage.
