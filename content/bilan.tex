\section{Bilan}

Comme nous l'avons vu, ce stage se place dans le contexte des communications au sein des clusters de calcul destinés au HPC.
Le stage se concentre sur les communications internes à un noeud par simplification pour faire une preuve de concept.
Nous nous demandons s'il est possible de plus avoir à faire de polling actif pour améliorer la réactivité grâce à des interruptions en espace utilisateur.

Pour les objectifs :
\begin{itemize}
  \item Nous avons vu le fonctionnement du mécanisme des interruptions en espace utilisateur et comment on peut l'utiliser pour les communications.
  \item Ensuite nous avons vu que la latence des \uintr{} est raisonnable comparée au passage par le noyau.
  \item Puis on a vu que nous avons intégré la progression à partir des handlers dans la bibliothèque \emph{NewMadeleine}
  mais il reste encore des choses à faire comme les interruptions pour la progression des envois,
  la correction des bugs, finaliser le support des gros paquets, et l'utilisation des interruptions dans le cas de progression multi-threadé avec \emph{Pioman}.
  \item Nous avons aussi vu comment nous avons désactivé le polling pour la réception et presque pour l'émission.
  \item On a montré que l'utilisation des \uintr{} permet bien l'\emph{overlap} pour la réception.
  Le comportement devrait être similaire pour l'émission.
\end{itemize}

Nous avons également fait une version du driver qui utilise les signaux, cette version n'était pas prévue initialement.

Le stage m'a beaucoup intéressé et j'ai appris pleins de choses autour du système, des bibliothèques de communications, du fonctionnement de la recherche, etc.
Mon travail a permis aussi de faire connaître en détail le nouveau mécanisme des \uintr{} aux équipes d'\emph{Inria} et d'\atos{}.

Le sujet de ce stage avait pour finalité de déboucher sur une thèse pour utiliser les \uintr{} au niveau du réseau \emph{BXI}.
Je continuerai donc en thèse ainsi je pourrai finir ce qu'il reste à faire et continuer les objectifs globaux.
