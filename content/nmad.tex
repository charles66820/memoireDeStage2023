\section{Intégration dans NewMadeleine}

\subsection{Présenté NewMadeleine details}

\begin{itemize}
  \item liste de progression recv
  \item liste de progression send
  \item p_pw
  \item post
  \item poll
  \item driver
  \item nm_schedule (appelé par nm_wait, ...)
  \item pioman
  \item existe : progression du nm_schedule ou de pioman vers le core_task
\end{itemize}

\subsection{Modification}

TODO: on désactive le poll avec les driver qui on des handler. Mais on fait toujours un poll qui fonctionne du premier coup.

\begin{itemize}
  \item ajout d'un driver sig_shm
  \item ajout d'un driver uintr_shm
  \item progression à partir du handler du driver vers les core_task
  \item problématiques de gestion des interruptions
  \begin{itemize}
    \item on ne peut pas faire d'attente dans les handler
    Les fonction doivent être async safe
    \item on ne peut pas utiliser d'allocateur donc il faut utiliser des p_pw déjà alloué
    \item pour avoir un p_pw disponible il faut faire progresser le communications
    \item la progresser à une partie critique qui naissaisite un verrou
    \item si on ne peut pas récupéré le core_lock (try_lock) il faut mettre à plus tard le trétement de l'interruption
    \item on utilise donc une file lock-free
    \item problématique des liste lock-free qui doivent être wait-free
    \begin{itemize}
      \item le principe d'une lfqueue est d'attente si quelqu'un d'autre modifie la file
      \item il faut donc une file wait-free
      \item j'ai fait de la biblio
      \item décrire les différante solutions
      \item solution d'Alexandre
    \end{itemize}
  \end{itemize}
  \item gestion des multiple interruptions qui s'écrase, (prob_any / pour les large si la progression à traiter le pipeline courant)
  \item quand est ce qu'on envois des interruption ?
  \begin{itemize}
    \item au moment ou on poste le premier paquet d'envois, pour indiquer au recepteur que des données sont disponible (le recepteur à toujours un paquet de reception posté)
    \item au moment ou une progression est fait du coté du recepteur, pour indiquer une reception a l'éméteur
    \item l'éméteur reçois une iterruption est détermine si l'émision est fini, petit paquet, ou si il faut envoyer la suite, paquet large.
  \end{itemize}
\end{itemize}

\subsection{Suite de tests}

Tous les tests ne passe pas ?

\subsection{Performances}

\subsubsection{Résultats avec attente active}

TODO: voire le sur coup des interruption

\subsubsection{Résultats recouvrement communication par du calcule}

TODO: courbe overlap reception