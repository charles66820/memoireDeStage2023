\section{Context}

\subsection{HPC}

\begin{itemize}
  \item simulation numérique
  \item utiliser sur super calculateur qui sont composé de plusieurs noeuds
  \item les noeuds sont connecté par un réseau haute performances dédié
  \item une communication est de l'ordre le la micro seconde
\end{itemize}

\subsection{OS bypass}

\begin{itemize}
  \item habituellement les périphériques sont programmé depuis le kernel.
  Ils utilise des interruptions pour avertir lors-ce que il y à un changement (e.g. un événement réseau).
  En HPC on évite de passer par le systeme car les context switch sont coûteux.
  Plusieurs µs (~3 sur la c4e).
  \item donc pour utiliser ce genre de réseau on programme directement la NIC depuis l'espace utilisateur
  \item 
\end{itemize}

\subsection{Polling}

\begin{itemize}
  \item consiste a scruté régulièrement une zone mémoire modifier par la NIC.
  \item cette zone mémoire peut-être une projection de la mémoire de la carte en RAM ou une zone de la RAM modifier par la carte.
  \item lors-ce que on scrute (Poll) la mémoire on peut déterminé l'états d'une communication réseau.
  \item pour faire progressé les communications il faut donc régulièrement poll
  \item a pour inconvenant une mauvaise réactivité et mobilise des resource de calcule pour faire des poll
\end{itemize}

\subsection{BXI}

\begin{itemize}
  \item les carte réseaux BXI (Bull eXascale Interconnect) sont dévelopé par Atos
  \item elle sont capable de faire des communications sans intervention du CPU % offload
  \item le CPU à juste à soumettre une commande et la carte s'occupe du reste
  \item pour savoir si la communication est fini l'application peut poll ou recevoir une interruption en provenance de la carte
  \item portal4
  \item 
\end{itemize}

\subsection{MPI}

\begin{itemize}
  \item présenté MPI
  \item les communications asynchrone
  \item la progression ce fait :
  \begin{itemize}
    \item au moment des appelle a la biblio (MPI_Isend, MPI_Irecv, MPI_test, MPI_wait).
    \item grâce à un thread dédier
    \item grâce à une politique d'utilisation des threads de façon opportunist (quand ils ne font pas de calcule).
    \item ...
  \end{itemize}
  \item 
\end{itemize}

\subsection{NewMadeleine}

\begin{itemize}
  \item NewMadeleine est une bibliothèque de communication...
  \item elle est basé sur un système de progression asynchrone...
  \item supporte différant type de communication grâce a un système de driver (portal4 pour BXI, ibverb pour IB, shm...)
  \item ...
\end{itemize}

\subsection{Travaux antérieur}

\begin{itemize}
  \item Travaux de Mathieu Barbe durant un stage en 2019 sur l'utilisation d'interruption pour transmettre des événements réseau.
  \item Ces travaux vis à diminué la latence du au fait de passer par un driver noyaux.
  \item Il parle dans les perspective de directement traitement des interruptions depuis l'espace utilisateur ce qui mène à mon stage.
\end{itemize}
