\section{Introduction}

J'ai effectué mon stage de 6 mois dans l'équipe-projet TADaaM du Centre Inria de l'université de Bordeaux.
Le sujet du stage à étais proposé par Alexandre Alexandre Denis (Inria) et Grégoire Pichon (Atos).
Ils on aussi encadré le stage aussi que Mathieu BARBE (Atos).

\subsection{Presentation Inria}

Inria est l'institut national français de recherche en sciences et technologies du numérique.
Compte plus de 3 900 chercheurs et ingénieurs au sein de 215 équipes-projets.
La plupart des centre sont communes avec une grande université.

% pas trop dans le detail mais juste globalement

\subsection{Presentation Atos (Eviden)}

Atos est un des leader international de la transformation numérique.
Atos couvre un champ large d'activité : le cloud, la cybersécurité, les services, le conseil, les supercalculateurs...
Atos compte 112 000 collaborateurs.
Atos est en restructuration pour ce séparé en 2 entités.
L'entités qui nous intéresse pour ce stage est Eviden qui englobe les activité autour des supercalculateurs et du HPC.
Cette restructuration est récente donc quand je parle d'Atos dans ce document je parle de la partie Eviden.

% pas trop dans le detail mais juste globalement

\subsection{Environnement de travail}

Mon environnement de travail c'est les locaux du Centre Inria de l'université de Bordeaux.
J'ai un bureau dans l'open space de l'équipe-projet TADaaM.
Je peut participé et assisté à des activités scientifique intéressantes (séminaire, soutenance de thèse, soutenance HDR, activité diverse...).
J'ai accès au salle de visio conférence, notamment pour les réunion de suivie de stage hebdomadaire.
Il y a aussi un Baby-foot, une cafeteria, une petite Médiathèque...

\subsection{Le cadre}

Du coté de l'équipe TADaaM de Inria l'objectif de l'équipe est de faire de la recherche.
Elle est composée de chercheurs, ingénieurs de recherche, post-doc, doctorant et stagiaire.
La culture informatique de l'équipe est l'utilisation des environnements linux, des logiciels open source,
des cluster de calcule HPC, le traitement des données et le système.
L'équipe à mis à ma disposition un ordinateur portable avec une station de travail brancher à un écran, à internet, à un clavier et une souris.
Inria donne aussi accès à un ensemble de services, boite email, service de communication Mattermost, service de visio Webex, intranet avec le menu de la cafeteria...
J'ai étais libre d'installer mon système d'exploitation, d'utilisé les outils informatique que je voulais...

Du coté d'Atos l'équipe BXI-LL (BXI Low Level) est une équipe don l'objectif est de fournir le support logiciels bas niveau pour les carte réseaux BXI.
Cela consiste à développé les drivers de la carte, le support Lustre\footnote{Lustre est un système de fichier parallèle et distribué. Il est utiliser pour l'I/O est ne fait pas partie de ce stage.} ainsi que l'interface logiciels Portal4.
Cette interface permet aux implémentations MPI d'utiliser le réseaux BXI.
Le personnel est composée d'ingénieurs, de doctorant et de stagiaire. /* TODO: vérifié */
La culture informatique de l'équipe est l'utilisation des cluster HPC, la programmation système et l'utilisation d'un système de sécurité PKI pour accédé aux ordinateurs et aux services interne.
Atos nous a donné accès à une machine avec deux processeur \intel{} Sapphire Rapids.
Pour accédé à cette machine ils nous on fourni un accès ssh et un compte VPN.
J'ai pu avoir tous les droits d'accès sur cette machine pour changé le noyaux du système.

\subsection{Presentation du plan}

Dans ce mémoire je vais commencé par vous présenté le context dans le quelle le stage ce place.
Je continuerai par les objectifs à long term et les objectifs du stage.
Je vous présenterai le nouveau mécanisme d'interruption et ce que j'ai fait avec.
Je présenterai l'intégration de ce mécanisme dans la bibliothèque de communication NewMadeleine et les tests que l'on à effectué.
Pour finir je ferai un bilan du stage et je parlerai de la suite.
