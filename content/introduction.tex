\section{Introduction}

J'ai effectué mon stage de 6 mois dans l'équipe-projet TADaaM d'Inria.
Le sujet du stage à étais proposé par Alexandre Alexandre Denis (Inria) et Grégoire Pichon (Atos).
Ils on aussi encadré le stage aussi que Mathieu BARBE (Atos).

\subsection{Presentation Inria}

Inria est TODO: here

% pas trop dans le detail mais juste globalement

\subsection{Presentation Atos (Eviden)}

% pas trop dans le detail mais juste globalement
en restructuration je parle donc d'Atos dans ce document

\subsection{Environnement de travail}

\begin{itemize}
  \item labo de recherche
  \item un bureau dans l'open space de l'équipe projet TADaaM.
  \item Salle de visio pour la réunion de suivie de stage toute le semaine
  \item Baby foot ?
  \item Cafeteria ?
  \item ... ?
\end{itemize}

\subsection{Le cadre}

Inria (TADaaM) :
\begin{itemize}
  \item objectifs du service : faire de la recherche...
  \item type de personnel : chercheur, post-doc, ingénieur, doctorant et stagiaire
  \item Culture informatique : HPC, système... ?
  \item Matériel et logiciels mis à disposition :
  \begin{itemize}
    \item un ordinateur portable
    \item un doc avec écran, clavier et souris
    \item service inria (email, mattermost, webex...)
  \end{itemize}
  \item contraintes : je suis libre pour la démarche, les outils...
\end{itemize}

Atos (BXI-LL) :
\begin{itemize}
  \item objectifs du service : développé le soft ba niveau pour la NIC BXI... (entre MPI et la carte) donc driver, lustre, portal...
  \item type de personnel : ingénieurs... ?
  \item Culture informatique : HPC, système... ?
  \item Matériel et logiciels mis à disposition :
  \begin{itemize}
    \item une machine avec 2 CPU SPR accessible via VPN
  \end{itemize}
  \item contraintes : je suis libre
\end{itemize}

\subsection{Presentation du plan}

Context dans le quelle le stage ce place puis les objectifs a long term et le objectifs du stage.
Le déroulement du stage avec les résultats. Pour finir un bilan et la suite.
TODO: 
