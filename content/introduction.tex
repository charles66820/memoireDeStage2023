\section{Introduction}

J'ai effectué mon stage de fin de master de six mois dans l'équipe-projet \emph{TADaaM} du Centre Inria de l'université de Bordeaux.
Le sujet du stage a été proposé par Alexandre Denis (Inria) et Grégoire Pichon (Atos).
Ils ont aussi encadré le stage avec Mathieu BARBE (Atos).

\subsection{Présentation Inria}

Inria est l'institut national français de recherche en sciences et technologies du numérique.
Il compte plus de 3 900 chercheurs et ingénieurs au sein de 215 équipes-projets.
La plupart des centres sont commun avec de grandes universités.

% pas trop dans le detail mais juste globalement

\subsection{Présentation Atos (Eviden)}

\atos{} est l'un des leaders internationaux de la transformation numérique.
Elle couvre un large éventail d'activités, notamment le cloud, la cybersécurité, les services transactionnels, le conseil, l'infogérance, le Big Data, les supercalculateurs, etc.
\atos{} compte 112 000 collaborateurs.
Actuellement, \atos{} est en restructuration afin de se séparer en deux entités.
L'entité qui nous intéresse pour ce stage est Eviden, qui englobe notamment les activités liées aux supercalculateurs et au HPC.
Cette restructuration est récente, donc lorsque nous évoquons \atos{} dans ce document, nous parlons de la partie Eviden.

\atos{} est le seul grand constructeur européen de supercalculateurs.
Elle est donc le leader européen et est bien positionnée mondialement.
Elle est présente dans le secteur du HPC depuis l'acquisition de la société \emph{Bull} en 2014.
\emph{Bull} était déjà impliquée dans le HPC depuis les années 2000 et en avait fait son coeur de métier.

La machine \emph{Leonardo} en Italie, construite par \atos{}, se classe quatrième au dernier Top500.
% cite https://www.top500.org/lists/top500/2023/06/
Ce classement évalue les machines les plus puissantes pour le HPC et est publié deux fois par an.

% pas trop dans le detail mais juste globalement

\subsection{Environnement de travail}

Mon environnement de travail se trouve dans les locaux du Centre Inria de l'université de Bordeaux.
Un bureau dans l'open space de l'équipe-projet \emph{TADaaM} a été mis à ma disposition.
Je peux participer et assister à des activités scientifiques intéressantes, telles que des séminaires, des soutenances de thèse, une soutenance HDR et d'autres activités diverses.
Nous avons également accès aux salles de visioconférence, notamment pour les réunions de suivi de stage hebdomadaires.
En plus, il y a un Baby-foot, une cafétéria, une petite médiathèque, etc.

\subsection{Le cadre}

Du côté de l'équipe \emph{TADaaM} d'Inria, l'objectif de l'équipe est de mener des recherches sur les sujets suivants :

\begin{itemize}
  \item Gestion des I/O (ordonnancement, bande passante...)
  \item Placement de processus
  \item Partitionnement de maillage (i.e. SCOTCH)
  \item Localité matérielle (i.e. hwloc)
  \item Optimisation des communications pour les réseaux haute performance, MPI (i.e. NewMadeleine)
\end{itemize}

Elle est composée de chercheurs, d'ingénieurs de recherche, de post-doc, de doctorants et de stagiaires.
Sa culture informatique est l'utilisation d'environnements Linux, de logiciels open source, de clusters de calcul HPC,
le traitement des données et le systèmes.
L'équipe a mis à ma disposition un ordinateur portable avec une station de travail reliée à un écran, à Internet, à un clavier et une souris.
Inria donne également l'accès à un ensemble de services, notamment une boîte e-mail, un service de communication (Mattermost), un service de visio (Webex) et un intranet avec le menu de la cafétéria, entre autres.
Nous sommes libres d'installer le système d'exploitation et d'utiliser les outils informatiques de notre choix.

Du côté d'\atos{} l'équipe BXI-LL (BXI Low Level) a pour objectif de fournir le support logiciel bas niveau pour les cartes réseaux \emph{BXI}.
Cela comprend le développement des pilotes de la carte, le support Lustre (un système de fichiers parallèle et distribué utilisé pour l'I/O, qui ne fait pas partie de ce stage), ainsi qu'une implémentation de l'interface logicielle \emph{Portal4}.
\emph{Portal4} est une interface logicielle qui permet une abstraction des opérations possibles sur les réseaux HPC.
Son but est de faciliter l'utilisation de ces opérations réseau avec les meilleures performances possibles pour les implémentations MPI. /* je suis pas sur de ma présentation de Portal4 */
L'implémentation d'\atos{} permet aux implémentations MPI d'utiliser le réseau \emph{BXI}.
Le personnel de l'équipe est composé d'ingénieurs, de doctorants et de stagiaires.
La culture informatique de l'équipe est l'utilisation de clusters HPC, la programmation système et l'utilisation d'un système de sécurité PKI pour accéder aux ordinateurs et aux services internes.
\atos{} nous a donné accès à une machine avec deux CPU \intel{} Sapphire Rapids.
Pour accéder à cette machine, ils nous ont fourni un accès SSH et un compte VPN.
Nous avons eu tous les droits d'accès sur cette machine pour changer le noyau du système.

\subsection{Présentation du plan}

Dans ce mémoire, nous commencerons par une présentation du contexte dans lequel le stage se place.
Ensuite, nous aborderons les objectifs à long terme ainsi que les objectifs du stage.
Nous continuerons en décrivant le nouveau mécanisme d'interruption en espace utilisateur et ce que nous en avons fait.
Ensuite, nous verrons l'intégration de ce mécanisme dans la bibliothèque de communication NewMadeleine, ainsi que les tests que nous avons effectués.
Pour finir, nous ferons un bilan du stage et évoquerons les travaux qui suivront.
