\section{Introduction}

J'ai effectué mon stage de 6 mois dans l'équipe-projet TADaaM du Centre Inria de l'université de Bordeaux.
Le sujet du stage à étais proposé par Alexandre Denis (Inria) et Grégoire Pichon (Atos).
Ils on aussi encadré le stage avec Mathieu BARBE (Atos).

\subsection{Présentation Inria}

Inria est l'institut national français de recherche en sciences et technologies du numérique.
Il compte plus de 3 900 chercheurs et ingénieurs au sein de 215 équipes-projets.
La plupart des centre sont commun avec une grande université.

% pas trop dans le detail mais juste globalement

\subsection{Présentation Atos (Eviden)}

Atos est un des leader international de la transformation numérique.
Elle couvre un champ large d'activité : le cloud, la cybersécurité, les services transactionnels, le conseil, l'infogérance, le Big Data, les supercalculateurs...
Atos compte 112 000 collaborateurs.
Atos est en restructuration pour ce séparé en 2 entités.
L'entités qui nous intéresse pour ce stage est Eviden qui englobe notamment les activité autour des supercalculateurs et du HPC.
Cette restructuration est récente donc quand Atos est évoqué dans ce document nous parlons de la partie Eviden.

% pas trop dans le detail mais juste globalement

\subsection{Environnement de travail}

Mon environnement de travail c'est les locaux du Centre Inria de l'université de Bordeaux.
On a mis à disposition un bureau dans l'open space de l'équipe-projet TADaaM.
Je peut participé et assisté à des activités scientifique intéressantes (séminaire, soutenance de thèse, soutenance HDR, activité diverse...).
Nous avons accès aux salle de visio conférence, notamment pour les réunion de suivie de stage hebdomadaire.
Il y a aussi un Baby-foot, une cafeteria, une petite Médiathèque...

\subsection{Le cadre}

Du coté de l'équipe TADaaM de Inria l'objectif de l'équipe est de faire de la recherche sur les sujets suivant :

\begin{itemize}
  \item Gestion des I/O (ordonnancement, bande passante...)
  \item Placement de processus
  \item Partitionnement de maillage (i.e. SCOTCH)
  \item Localité materiel (i.e. hwloc)
  \item Optimisation des communications pour les réseaux haute performance, MPI (i.e. NewMadeleine)
\end{itemize}

Elle est composée de chercheurs, ingénieurs de recherche, post-doc, doctorant et stagiaire.
La culture informatique de l'équipe est l'utilisation des environnements linux, des logiciels open source,
des cluster de calcule HPC, le traitement des données et le système.
L'équipe à mis à ma disposition un ordinateur portable avec une station de travail brancher à un écran, à internet, à un clavier et une souris.
Inria donne aussi accès à un ensemble de services, boite email, service de communication Mattermost, service de visio Webex, intranet avec le menu de la cafeteria...
Nous somme libre d'installer le système d'exploitation et d'utilisé les outils informatique que que l'on veut.

Du coté d'Atos l'équipe BXI-LL (BXI Low Level) est une équipe don l'objectif est de fournir le support logiciels bas niveau pour les carte réseaux BXI.
Cela consiste à développé les drivers de la carte, le support Lustre -- un système de fichier parallèle et distribué qui est utiliser pour l'I/O est ne fait pas partie de ce stage -- ainsi que l'interface logiciels Portal4.
Cette interface permet aux implémentations MPI d'utiliser le réseaux BXI.
Le personnel est composée d'ingénieurs, de doctorant et de stagiaire. /* TODO: vérifié */
La culture informatique de l'équipe est l'utilisation des cluster HPC, la programmation système et l'utilisation d'un système de sécurité PKI pour accédé aux ordinateurs et aux services interne.
Atos nous a donné accès à une machine avec deux CPU \intel{} Sapphire Rapids.
Pour accédé à cette machine ils nous on fourni un accès ssh et un compte VPN.
Nous avons eu tous les droits d'accès sur cette machine pour changé le noyaux du système.

\subsection{Présentation du plan}

Dans ce mémoire nous allons commencé par une présentation du context dans le quelle le stage ce place.
Puis nous continuerons avec les objectifs à long term et les objectifs du stage.
Nous verrons ensuite le nouveau mécanisme d'interruption et ce que nous avons fait avec.
Nous continuerons par la présentation de l'intégration de ce mécanisme dans la bibliothèque de communication NewMadeleine et les testes que l'on à effectué.
Pour finir un bilan du stage ainsi que les travaux qui vont suivre.
