\section{Exploration de uintr}

\subsection{Prérequis est accès}

\begin{itemize}
  \item difficulté d'accès au processeur...
  \item J'ai eu accès au processeur ~2 mois 1/2 après le début du stage
  \item J'ai donc eu accès à une machine avec 2 CPU SPR... VPN...
  \item il faut une version patché du noyaux linux pour utilisé c'est nouvelle interruption
  Ce patch n'est pas disponible en mainline linux.
  J'ai donc télécharger le patch fourni par intel, je l'ai compiler puis je l'ai installer sur la machine
  \item Pour utiliser les nouvelle instructions il faut une version récente de GCC... (version).
  Pas disponible avec LLVM-Clang
\end{itemize}

\subsection{Fonctionnement des uintr}

\subsubsection{Détail}

\begin{itemize}
  \item 5 instructions
  \item registre MSR pour l'init
\end{itemize}

\subsubsection{Capacité présente et futur}

\begin{itemize}
  \item process => process
  \item kernel => process
  \item device => process
  \item au niveaux des thread...
  \item masquage
  \item le thread dois être en ring-3
  \item peut fonctionné si le thread est endormi mais j'ai pas testé car ...
\end{itemize}

\subsubsection{Example}

\begin{itemize}
  \item example avec schéma
\end{itemize}

\subsubsection{Partage du FD}

\begin{itemize}
  \item Pour mes testes "jouer" j'ai utiliser \verb|uintr_register_self()|
  \item pipe / socket / URL pour NewMadeleine (on en reparle en section XXX)
  \item \verb|pidfd_getfd|
\end{itemize}

\subsubsection{+}

\begin{itemize}
  \item comparé au signaux ?
  \item TODO ...
\end{itemize}

\subsection{Tests du mécanisme}

\begin{itemize}
  \item entre process
  \item entre threads
  \item avec alt stack
  \item test d'écrasement des interruptions
  \item avec binding
  \item avec turbo boost
\end{itemize}

\subsection{Correction du patch pour l'appel system uintr_alt_stack}

TODO

\subsection{Performances}

TODO

\begin{itemize}
  \item Dans un premier temps j'ai pas bind les thread par pu donc j'avais de mauvais résultats.
  \item J'ai fait différents bind (on vois que c'est similaire sauf quand on passe entre 2 NUMA)
  \item si on monte la frequence c'est mieux (turbo boost)
  \item 
\end{itemize}
